\documentclass[]{article}
\usepackage{lmodern}
\usepackage{amssymb,amsmath}
\usepackage{ifxetex,ifluatex}
\usepackage{fixltx2e} % provides \textsubscript
\ifnum 0\ifxetex 1\fi\ifluatex 1\fi=0 % if pdftex
  \usepackage[T1]{fontenc}
  \usepackage[utf8]{inputenc}
\else % if luatex or xelatex
  \ifxetex
    \usepackage{mathspec}
  \else
    \usepackage{fontspec}
  \fi
  \defaultfontfeatures{Ligatures=TeX,Scale=MatchLowercase}
\fi
% use upquote if available, for straight quotes in verbatim environments
\IfFileExists{upquote.sty}{\usepackage{upquote}}{}
% use microtype if available
\IfFileExists{microtype.sty}{%
\usepackage{microtype}
\UseMicrotypeSet[protrusion]{basicmath} % disable protrusion for tt fonts
}{}
\usepackage[margin=1in]{geometry}
\usepackage{hyperref}
\hypersetup{unicode=true,
            pdftitle={Problem set 1},
            pdfauthor={Jason Parker},
            pdfborder={0 0 0},
            breaklinks=true}
\urlstyle{same}  % don't use monospace font for urls
\usepackage{color}
\usepackage{fancyvrb}
\newcommand{\VerbBar}{|}
\newcommand{\VERB}{\Verb[commandchars=\\\{\}]}
\DefineVerbatimEnvironment{Highlighting}{Verbatim}{commandchars=\\\{\}}
% Add ',fontsize=\small' for more characters per line
\usepackage{framed}
\definecolor{shadecolor}{RGB}{248,248,248}
\newenvironment{Shaded}{\begin{snugshade}}{\end{snugshade}}
\newcommand{\KeywordTok}[1]{\textcolor[rgb]{0.13,0.29,0.53}{\textbf{#1}}}
\newcommand{\DataTypeTok}[1]{\textcolor[rgb]{0.13,0.29,0.53}{#1}}
\newcommand{\DecValTok}[1]{\textcolor[rgb]{0.00,0.00,0.81}{#1}}
\newcommand{\BaseNTok}[1]{\textcolor[rgb]{0.00,0.00,0.81}{#1}}
\newcommand{\FloatTok}[1]{\textcolor[rgb]{0.00,0.00,0.81}{#1}}
\newcommand{\ConstantTok}[1]{\textcolor[rgb]{0.00,0.00,0.00}{#1}}
\newcommand{\CharTok}[1]{\textcolor[rgb]{0.31,0.60,0.02}{#1}}
\newcommand{\SpecialCharTok}[1]{\textcolor[rgb]{0.00,0.00,0.00}{#1}}
\newcommand{\StringTok}[1]{\textcolor[rgb]{0.31,0.60,0.02}{#1}}
\newcommand{\VerbatimStringTok}[1]{\textcolor[rgb]{0.31,0.60,0.02}{#1}}
\newcommand{\SpecialStringTok}[1]{\textcolor[rgb]{0.31,0.60,0.02}{#1}}
\newcommand{\ImportTok}[1]{#1}
\newcommand{\CommentTok}[1]{\textcolor[rgb]{0.56,0.35,0.01}{\textit{#1}}}
\newcommand{\DocumentationTok}[1]{\textcolor[rgb]{0.56,0.35,0.01}{\textbf{\textit{#1}}}}
\newcommand{\AnnotationTok}[1]{\textcolor[rgb]{0.56,0.35,0.01}{\textbf{\textit{#1}}}}
\newcommand{\CommentVarTok}[1]{\textcolor[rgb]{0.56,0.35,0.01}{\textbf{\textit{#1}}}}
\newcommand{\OtherTok}[1]{\textcolor[rgb]{0.56,0.35,0.01}{#1}}
\newcommand{\FunctionTok}[1]{\textcolor[rgb]{0.00,0.00,0.00}{#1}}
\newcommand{\VariableTok}[1]{\textcolor[rgb]{0.00,0.00,0.00}{#1}}
\newcommand{\ControlFlowTok}[1]{\textcolor[rgb]{0.13,0.29,0.53}{\textbf{#1}}}
\newcommand{\OperatorTok}[1]{\textcolor[rgb]{0.81,0.36,0.00}{\textbf{#1}}}
\newcommand{\BuiltInTok}[1]{#1}
\newcommand{\ExtensionTok}[1]{#1}
\newcommand{\PreprocessorTok}[1]{\textcolor[rgb]{0.56,0.35,0.01}{\textit{#1}}}
\newcommand{\AttributeTok}[1]{\textcolor[rgb]{0.77,0.63,0.00}{#1}}
\newcommand{\RegionMarkerTok}[1]{#1}
\newcommand{\InformationTok}[1]{\textcolor[rgb]{0.56,0.35,0.01}{\textbf{\textit{#1}}}}
\newcommand{\WarningTok}[1]{\textcolor[rgb]{0.56,0.35,0.01}{\textbf{\textit{#1}}}}
\newcommand{\AlertTok}[1]{\textcolor[rgb]{0.94,0.16,0.16}{#1}}
\newcommand{\ErrorTok}[1]{\textcolor[rgb]{0.64,0.00,0.00}{\textbf{#1}}}
\newcommand{\NormalTok}[1]{#1}
\usepackage{graphicx,grffile}
\makeatletter
\def\maxwidth{\ifdim\Gin@nat@width>\linewidth\linewidth\else\Gin@nat@width\fi}
\def\maxheight{\ifdim\Gin@nat@height>\textheight\textheight\else\Gin@nat@height\fi}
\makeatother
% Scale images if necessary, so that they will not overflow the page
% margins by default, and it is still possible to overwrite the defaults
% using explicit options in \includegraphics[width, height, ...]{}
\setkeys{Gin}{width=\maxwidth,height=\maxheight,keepaspectratio}
\IfFileExists{parskip.sty}{%
\usepackage{parskip}
}{% else
\setlength{\parindent}{0pt}
\setlength{\parskip}{6pt plus 2pt minus 1pt}
}
\setlength{\emergencystretch}{3em}  % prevent overfull lines
\providecommand{\tightlist}{%
  \setlength{\itemsep}{0pt}\setlength{\parskip}{0pt}}
\setcounter{secnumdepth}{0}
% Redefines (sub)paragraphs to behave more like sections
\ifx\paragraph\undefined\else
\let\oldparagraph\paragraph
\renewcommand{\paragraph}[1]{\oldparagraph{#1}\mbox{}}
\fi
\ifx\subparagraph\undefined\else
\let\oldsubparagraph\subparagraph
\renewcommand{\subparagraph}[1]{\oldsubparagraph{#1}\mbox{}}
\fi

%%% Use protect on footnotes to avoid problems with footnotes in titles
\let\rmarkdownfootnote\footnote%
\def\footnote{\protect\rmarkdownfootnote}

%%% Change title format to be more compact
\usepackage{titling}

% Create subtitle command for use in maketitle
\providecommand{\subtitle}[1]{
  \posttitle{
    \begin{center}\large#1\end{center}
    }
}

\setlength{\droptitle}{-2em}

  \title{Problem set 1}
    \pretitle{\vspace{\droptitle}\centering\huge}
  \posttitle{\par}
    \author{Jason Parker}
    \preauthor{\centering\large\emph}
  \postauthor{\par}
      \predate{\centering\large\emph}
  \postdate{\par}
    \date{Due Sept 17th}


\begin{document}
\maketitle

\section{Brooks Questions:}\label{brooks-questions}

Use R for the Chris Brooks problems. You can find any of the data sets
on the book
\href{https://www.cambridge.org/us/academic/subjects/economics/finance/introductory-econometrics-finance-4th-edition}{website}.

Do questions 3.5-3.9.

Do questions 4.1-4.6.

\section{Supplementary Questions:}\label{supplementary-questions}

In the supplementary questions, the data sets are available as tables in
the \texttt{wooldridge2.db} file.

\subsection{Question 1}\label{question-1}

An idempotent matrix \(A\) is one where \(AA=A\). In this question, use
that \(\hat\beta=(X'X)^{-1}X'y\) and \(y=X\beta+e\). Define
\(P=X(X'X)^{-1}X'\) and \(M=I_n-P_X\). Show the following:

\begin{enumerate}
\def\labelenumi{\roman{enumi}.}
\tightlist
\item
  \(P\) is idempotent \(P\) = \(X(X'X)^{-1}X'\)*\(X(X'X)^{-1}X'\)
\item
  \(M\) is idempotent
\item
  \(\hat{y}=P y\)
\item
  \(\hat{e}=M y\)
\item
  \(y=P y+M y\)
\item
  \(\hat{y}\perp\hat{e}\)
\end{enumerate}

\subsection{Question 2}\label{question-2}

The data in \texttt{meap01} table are for the state of Michigan in the
year \(2001\). Use these data to answer the following questions.

\begin{enumerate}
\def\labelenumi{\arabic{enumi}.}
\tightlist
\item
  Find the largest and smallest values of math4. Does the range make
  sense? Explain.
\end{enumerate}

\begin{Shaded}
\begin{Highlighting}[]
\NormalTok{meap01 <-}\StringTok{ }\KeywordTok{wpull}\NormalTok{(}\StringTok{'meap01'}\NormalTok{)}
\end{Highlighting}
\end{Shaded}

\begin{verbatim}
##   index variable.name  type format
## 1     0         dcode float  %9.0g
## 2     1         bcode   int  %9.0g
## 3     2         math4 float  %9.0g
## 4     3         read4 float  %9.0g
## 5     4         lunch float  %9.0g
## 6     5        enroll   int  %9.0g
## 7     6        expend float  %9.0g
## 8     7         exppp float  %9.0g
##                                  variable.label
## 1                                 district code
## 2                                 building code
## 3       % students satisfactory, 4th grade math
## 4    % students satisfactory, 4th grade reading
## 5 % students eligible for free or reduced lunch
## 6                             school enrollment
## 7                             total spending, $
## 8         expenditures per pupil: expend/enroll
\end{verbatim}

\begin{Shaded}
\begin{Highlighting}[]
\KeywordTok{summary}\NormalTok{(meap01}\OperatorTok{$}\NormalTok{math4)}
\end{Highlighting}
\end{Shaded}

\begin{verbatim}
##    Min. 1st Qu.  Median    Mean 3rd Qu.    Max. 
##    0.00   61.60   76.40   71.91   87.00  100.00
\end{verbatim}

\begin{Shaded}
\begin{Highlighting}[]
\KeywordTok{paste}\NormalTok{(}\StringTok{"The range do make sense since math scores should range between 0-100"}\NormalTok{)}
\end{Highlighting}
\end{Shaded}

\begin{verbatim}
## [1] "The range do make sense since math scores should range between 0-100"
\end{verbatim}

\begin{enumerate}
\def\labelenumi{\arabic{enumi}.}
\setcounter{enumi}{1}
\tightlist
\item
  How many schools have a perfect pass rate on the math test? What
  percentage is this of the total sample?
\end{enumerate}

\begin{Shaded}
\begin{Highlighting}[]
\NormalTok{x =}\StringTok{ }\KeywordTok{sum}\NormalTok{(meap01}\OperatorTok{$}\NormalTok{math4 }\OperatorTok{>}\StringTok{ }\DecValTok{99}\NormalTok{)}
\NormalTok{y =}\StringTok{ }\NormalTok{(}\KeywordTok{sum}\NormalTok{(meap01}\OperatorTok{$}\NormalTok{math4 }\OperatorTok{>}\StringTok{ }\DecValTok{99}\NormalTok{) }\OperatorTok{/}\StringTok{ }\KeywordTok{length}\NormalTok{(meap01}\OperatorTok{$}\NormalTok{index)) }\OperatorTok{*}\StringTok{ }\DecValTok{100} 
\KeywordTok{paste}\NormalTok{(}\StringTok{"Schools with perfect pass rate: "}\NormalTok{, x)}
\end{Highlighting}
\end{Shaded}

\begin{verbatim}
## [1] "Schools with perfect pass rate:  38"
\end{verbatim}

\begin{Shaded}
\begin{Highlighting}[]
\KeywordTok{paste}\NormalTok{(}\StringTok{"% of total sample is: "}\NormalTok{, y,}\StringTok{"%"}\NormalTok{)}
\end{Highlighting}
\end{Shaded}

\begin{verbatim}
## [1] "% of total sample is:  2.08447613823368 %"
\end{verbatim}

\begin{enumerate}
\def\labelenumi{\arabic{enumi}.}
\setcounter{enumi}{2}
\tightlist
\item
  How many schools have math pass rates of exactly \(50\%\)?
\end{enumerate}

\begin{Shaded}
\begin{Highlighting}[]
\KeywordTok{sum}\NormalTok{(meap01}\OperatorTok{$}\NormalTok{math4 }\OperatorTok{==}\StringTok{ }\DecValTok{50}\NormalTok{)}
\end{Highlighting}
\end{Shaded}

\begin{verbatim}
## [1] 17
\end{verbatim}

\begin{enumerate}
\def\labelenumi{\arabic{enumi}.}
\setcounter{enumi}{3}
\tightlist
\item
  Compare the average pass rates for the math and reading scores. Which
  test is harder to pass?
\end{enumerate}

\begin{Shaded}
\begin{Highlighting}[]
\NormalTok{math =}\StringTok{ }\KeywordTok{mean}\NormalTok{(meap01}\OperatorTok{$}\NormalTok{math4,}\DataTypeTok{trim =} \FloatTok{0.5}\NormalTok{)}
\NormalTok{read =}\StringTok{ }\KeywordTok{mean}\NormalTok{(meap01}\OperatorTok{$}\NormalTok{read4,}\DataTypeTok{trim =} \FloatTok{0.5}\NormalTok{)}
\KeywordTok{paste}\NormalTok{(math, }\StringTok{">"}\NormalTok{,read ,math }\OperatorTok{>}\StringTok{ }\NormalTok{read, }\StringTok{"Reading is harder then math"}\NormalTok{)}
\end{Highlighting}
\end{Shaded}

\begin{verbatim}
## [1] "76.4 > 62.7 TRUE Reading is harder then math"
\end{verbatim}

\begin{enumerate}
\def\labelenumi{\arabic{enumi}.}
\setcounter{enumi}{4}
\tightlist
\item
  Find the correlation between math4 and read4. What do you conclude?
\end{enumerate}

\begin{Shaded}
\begin{Highlighting}[]
\KeywordTok{cor}\NormalTok{(meap01}\OperatorTok{$}\NormalTok{math4, meap01}\OperatorTok{$}\NormalTok{read4 )}
\end{Highlighting}
\end{Shaded}

\begin{verbatim}
## [1] 0.8427281
\end{verbatim}

\begin{Shaded}
\begin{Highlighting}[]
\KeywordTok{paste}\NormalTok{(}\StringTok{"Strong linear relationship since a high pass rate on math corrs with high pass rate on reading."}\NormalTok{)}
\end{Highlighting}
\end{Shaded}

\begin{verbatim}
## [1] "Strong linear relationship since a high pass rate on math corrs with high pass rate on reading."
\end{verbatim}

\begin{enumerate}
\def\labelenumi{\arabic{enumi}.}
\setcounter{enumi}{5}
\tightlist
\item
  The variable exppp is expenditure per pupil. Find the average of exppp
  along with its standard deviation. Would you say there is wide
  variation in per pupil spending?
\end{enumerate}

\begin{Shaded}
\begin{Highlighting}[]
\NormalTok{u =}\StringTok{ }\KeywordTok{mean}\NormalTok{(meap01}\OperatorTok{$}\NormalTok{exppp)}
\NormalTok{std =}\StringTok{ }\KeywordTok{sd}\NormalTok{(meap01}\OperatorTok{$}\NormalTok{exppp) }\OperatorTok{/}\StringTok{ }\KeywordTok{mean}\NormalTok{(meap01}\OperatorTok{$}\NormalTok{exppp)}
\KeywordTok{mean}\NormalTok{(meap01}\OperatorTok{$}\NormalTok{exppp) }\OperatorTok{+}\StringTok{ }\KeywordTok{sd}\NormalTok{(meap01}\OperatorTok{$}\NormalTok{exppp) }\OperatorTok{/}\StringTok{ }\KeywordTok{mean}\NormalTok{(meap01}\OperatorTok{$}\NormalTok{exppp)}
\end{Highlighting}
\end{Shaded}

\begin{verbatim}
## [1] 5195.076
\end{verbatim}

\begin{Shaded}
\begin{Highlighting}[]
\KeywordTok{c}\NormalTok{(u}\OperatorTok{-}\DecValTok{2}\OperatorTok{*}\NormalTok{std, u}\OperatorTok{+}\DecValTok{2}\OperatorTok{*}\NormalTok{std)}
\end{Highlighting}
\end{Shaded}

\begin{verbatim}
## [1] 5194.445 5195.286
\end{verbatim}

\begin{Shaded}
\begin{Highlighting}[]
\KeywordTok{paste}\NormalTok{(}\StringTok{"Not much variation in per pupil spending!"}\NormalTok{)}
\end{Highlighting}
\end{Shaded}

\begin{verbatim}
## [1] "Not much variation in per pupil spending!"
\end{verbatim}

\begin{enumerate}
\def\labelenumi{\arabic{enumi}.}
\setcounter{enumi}{6}
\tightlist
\item
  Suppose School A spends \(\$6000\) per student and School B spends
  \(\$5500\) per student. By what percentage does School A's spending
  exceed School B's? Compare this to \(100*[\ln(6000) - \ln(5500)]\),
  which is the approximation percentage difference based on the
  difference in the natural logs.
\end{enumerate}

\begin{Shaded}
\begin{Highlighting}[]
\NormalTok{percent =((}\DecValTok{6000}\OperatorTok{-}\DecValTok{5500}\NormalTok{)}\OperatorTok{/}\DecValTok{5500}\NormalTok{)}\OperatorTok{*}\DecValTok{100}
\NormalTok{log_diff =}\StringTok{ }\DecValTok{100} \OperatorTok{*}\StringTok{ }\NormalTok{(}\KeywordTok{log}\NormalTok{(}\DecValTok{6000}\OperatorTok{/}\DecValTok{5500}\NormalTok{))}
\KeywordTok{paste}\NormalTok{(percent, }\StringTok{">"}\NormalTok{, log_diff, }\StringTok{"Log difference is always more conservative than simple percentage difference."}\NormalTok{)}
\end{Highlighting}
\end{Shaded}

\begin{verbatim}
## [1] "9.09090909090909 > 8.70113769896297 Log difference is always more conservative than simple percentage difference."
\end{verbatim}

\subsection{Question 3}\label{question-3}

The data in \texttt{401k} are a subset of data analyzed by Papke
(\(1995\)) to study the relationship between participation in a
\(401(k)\) pension plan and the generosity of the plan. The variable
prate is the percentage of eligible workers with an active account; this
is the variable we would like to explain. The measure of generosity is
the plan match rate, mrate. This variable gives the average amount the
firm contributes to each worker's plan for each \(\$1\) contribution by
the worker. For example, if \(mrate=0.50\), then a \(\$1\) contribution
by the worker is matched by a \(50?\) contribution by the firm.

\begin{enumerate}
\def\labelenumi{\arabic{enumi}.}
\tightlist
\item
  Find the average participation rate and the average match rate in the
  sample of plans.
\end{enumerate}

\begin{Shaded}
\begin{Highlighting}[]
\NormalTok{four01k =}\StringTok{ }\KeywordTok{wpull}\NormalTok{(}\StringTok{'401k'}\NormalTok{)}
\end{Highlighting}
\end{Shaded}

\begin{verbatim}
##   index variable.name  type format                  variable.label
## 1     0         prate float  %7.0g     participation rate, percent
## 2     1         mrate float  %7.0g            401k plan match rate
## 3     2       totpart float  %7.0g         total 401k participants
## 4     3        totelg float  %7.0g    total eligible for 401k plan
## 5     4           age  byte  %7.0g                age of 401k plan
## 6     5        totemp float  %7.0g  total number of firm employees
## 7     6          sole  byte  %7.0g = 1 if 401k is firm's sole plan
## 8     7       ltotemp float  %9.0g                   log of totemp
\end{verbatim}

\begin{Shaded}
\begin{Highlighting}[]
\KeywordTok{c}\NormalTok{(}\StringTok{"Avg participation rate"}\NormalTok{,}\StringTok{"Avg match rate"}\NormalTok{)}
\end{Highlighting}
\end{Shaded}

\begin{verbatim}
## [1] "Avg participation rate" "Avg match rate"
\end{verbatim}

\begin{Shaded}
\begin{Highlighting}[]
\KeywordTok{c}\NormalTok{(}\KeywordTok{mean}\NormalTok{(four01k}\OperatorTok{$}\NormalTok{prate),}\KeywordTok{mean}\NormalTok{(four01k}\OperatorTok{$}\NormalTok{mrate))}
\end{Highlighting}
\end{Shaded}

\begin{verbatim}
## [1] 87.3629074  0.7315124
\end{verbatim}

\begin{enumerate}
\def\labelenumi{\arabic{enumi}.}
\setcounter{enumi}{1}
\item
  Now, estimate the simple regression equation:
  \[\hat{prate} = \hat{\beta}_0+\hat{\beta}_1 mrate,\] and report the
  results along with the sample size and R-squared.
\item
  Interpret the intercept in your equation. Interpret the coefficient on
  \texttt{mrate}.
\item
  Find the predicted \texttt{prate} when \(mrate=3.5\). Is this a
  reasonable prediction? Explain what is happening here.
\item
  How much of the variation in \texttt{prate} is explained by
  \texttt{mrate}?
\end{enumerate}

\subsection{Question 4}\label{question-4}

The data set in \texttt{ceosal2} contains information on chief executive
officers for U.S. corporations. The variable salary is annual
compensation, in thousands of dollars, and ceoten is prior number of
years as company CEO.

\begin{enumerate}
\def\labelenumi{\arabic{enumi}.}
\tightlist
\item
  Find the average salary and the average tenure in the sample.
\item
  How many CEOs are in their first year as CEO (that is, \(ceoten=0\))?
  What is the longest tenure as a CEO?
\item
  Estimate the simple regression model
  \[\ln[salary]=\beta_0+\beta_1 ceoten+u\] and report your results in
  the usual form. What is the (approximate) predicted percentage
  increase in salary given one more year as a CEO?
\end{enumerate}

\subsection{Question 5}\label{question-5}

Use the data in \texttt{wage2} to estimate a simple regression
explaining monthly salary (\texttt{wage}) in terms of IQ score
(\texttt{IQ}).

\begin{enumerate}
\def\labelenumi{\arabic{enumi}.}
\tightlist
\item
  Find the average salary and average IQ in the sample. What is the
  sample standard deviation of IQ? (IQ scores are standardized so that
  the average in the population is \(100\) with a standard deviation
  equal to \(15\).)
\item
  Estimate a simple regression model where a one-point increase in IQ
  changes wage by a constant dollar amount. Use this model to find the
  predicted increase in wage for an increase in IQ of \(15\) points.
  Does IQ explain most of the variation in wage?
\item
  Now, estimate a model where each one-point increase in IQ has the same
  percentage effect on wage. If IQ increases by \(15\) points, what is
  the approximate percentage increase in predicted wage?
\end{enumerate}

\subsection{Question 6}\label{question-6}

Using the \texttt{meap93} data, we want to explore the relationship
between the math pass rate (\texttt{math10}) and spending per student
(\texttt{expend}).

\begin{enumerate}
\def\labelenumi{\arabic{enumi}.}
\tightlist
\item
  Do you think each additional dollar spent has the same effect on the
  pass rate, or does a diminishing effect seem more appropriate?
  Explain.
\item
  In the population model, \[math10 = \beta_0+\beta_1 \ln[expend] + u\]
  argue that \(\beta_1/10\) is the percentage point change in math10
  given a \(10\%\) increase in expend.
\item
  Estimate this model. Report the estimated equation in the usual way,
  including the sample size and R-squared.
\item
  How big is the estimated spending effect? Namely, if spending
  increases by \(10\%\), what is the estimated percentage point increase
  in \texttt{math10}?
\item
  One might worry that regression analysis can produce fitted values for
  \texttt{math10} that are greater than \(100\). Why is this not much of
  a worry in this data set?
\end{enumerate}

\subsection{Question 7}\label{question-7}

Use the data in \texttt{hprice1} to estimate the model
\[price=\beta_0+\beta_1 sqrft+\beta_2 bdrms + u,\] where price is the
house price measured in thousands of dollars.

\begin{enumerate}
\def\labelenumi{\arabic{enumi}.}
\tightlist
\item
  Write out the results in equation form.
\item
  What is the estimated increase in price for a house with one more
  bedroom, holding square footage constant?
\item
  What is the estimated increase in price for a house with an additional
  bedroom that is \(140\) square feet in size? Compare this to your
  answer from above.
\item
  What percentage of the variation in price is explained by square
  footage and number of bedrooms?
\item
  The first house in the sample has \(sqrft=2438\) and \(bdrms=4\). Find
  the predicted selling price for this house from the OLS regression
  line.
\item
  The actual selling price of the first house in the sample was \$300000
  (so \(price=300\)). Find the residual for this house. Does it suggest
  that the buyer underpaid or overpaid for the house?
\end{enumerate}

\subsection{Question 8}\label{question-8}

The file \texttt{ceosal2} contains data on \(177\) chief executive
officers and can be used to examine the effects of firm performance on
CEO salary.

\begin{enumerate}
\def\labelenumi{\arabic{enumi}.}
\tightlist
\item
  Estimate a model relating annual salary to firm sales and market
  value. Make the model of the constant elasticity variety for both
  independent variables. Write the results out in equation form.
\item
  Add profits to the model. Why can this variable not be included in
  logarithmic form? Would you say that these firm performance variables
  explain most of the variation in CEO salaries?
\item
  Now also add the variable ceoten to the model. What is the estimated
  percentage return for another year of CEO tenure, holding other
  factors fixed?
\item
  Find the sample correlation coefficient between the variables
  \(log(mktval)\) and \(profits\). Are these variables highly
  correlated? What does this say about the OLS estimators?
\end{enumerate}

\subsection{Question 9}\label{question-9}

Use the data in \texttt{attend} for this exercise. Create the variable
\texttt{atndrte} which is \(attend/32\) because there were \(32\)
classes.

\begin{enumerate}
\def\labelenumi{\arabic{enumi}.}
\tightlist
\item
  Obtain the minimum, maximum, and average values for the variables
  \texttt{atndrte}, \texttt{priGPA}, and \texttt{ACT}.
\item
  Estimate the model \[atndrte=\beta_0+\beta_1 GPA + \beta_2 ACT + u\]
  and write the results in equation form. Interpret the intercept. Does
  it have a useful meaning?
\item
  Discuss the estimated slope coefficients. Are there any surprises?
\item
  What is the predicted \texttt{atndrte} if \(priGPA=3.65\) and
  \(ACT=20\)? What do you make of this result? Are there any students in
  the sample with these values of the explanatory variables?
\item
  If Student A has \(priGPA=3.1\) and \(ACT=21\) and Student B has
  \(priGPA=2.1\) and \(ACT=26\), what is the predicted difference in
  their attendance rates?
\end{enumerate}

\subsection{Question 10}\label{question-10}

Use the data in \texttt{htv} to answer this question. The data set
includes information on wages, education, parents' education, and
several other variables for \(1,230\) working men in \(1991\).

\begin{enumerate}
\def\labelenumi{\arabic{enumi}.}
\tightlist
\item
  What is the range of the educ variable in the sample? What percentage
  of men completed 12th grade but no higher grade? Do the men or their
  parents have, on average, higher levels of education?
\item
  Estimate the regression model
  \[educ=\beta_0+\beta_1 motheduc+\beta_2 fatheduc+u\] by OLS and report
  the results in the usual form. How much sample variation in educ is
  explained by parents' education? Interpret the coefficient on
  motheduc.
\item
  Add the variable abil (a measure of cognitive ability) to the
  regression above, and report the results in equation form. Does
  \texttt{ability} help to explain variations in education, even after
  controlling for parents' education? Explain.
\item
  Now estimate an equation where abil appears in quadratic form:
  \[educ=\beta_0+\beta_1 motheduc+\beta_2 fatheduc+\beta_3 abil+\beta_4{abil}^2+u.\]
  With the estimated coefficients on ability, use calculus to find the
  value of abil where educ is minimized. (The other coefficients and
  values of parents' education variables have no effect; we are holding
  parents' education fixed.) Notice that abil is measured so that
  negative values are permissible. You might also verify that the second
  derivative is positive so that you do indeed have a minimum.
\item
  Argue that only a small fraction of men in the sample have
  \texttt{ability} less than the value calculated above. Why is this
  important?
\item
  Use the estimates above to plot the relationship beween the predicted
  education and abil. Let \texttt{motheduc} and \texttt{fatheduc} have
  their average values in the sample, \(12.18\) and \(12.45\),
  respectively.
\end{enumerate}


\end{document}
